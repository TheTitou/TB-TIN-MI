% Francais
% Cette thèse, en complément à une thèse réalisée en parrallèle, a été réalisée dans le but de caractériser le site 
% d'observation astronomique de IRSOL (Tessin).
% L'objectif est de quantifier la perturbations atmosphérique du site d'observation.
% L'outil qui a été sélectionné est un système appelé DIMM (Differential image motion monitor). 
% Le principe est simple. Il suffit de masquer une image en deux pour en faire une mesure différentielle. 
% La mesure différentielle permet d'annuler les perturbations extérieures qui pourraient apparaîtres autour du télescope 
% (par exemple les mouvements et vibrations)
% La perturbation atmosphérique induira des mouvements de cette image et le seeing se qualifiera grace à l'écart type de la mesure.
% Le résultat pourra être exprimé avec l'écart type de l'angle d'incidence des faisceau ($\sigma$) ou par le paramètre de Fried ($r_0$) correspondant.
% Le projet a été mené a bien en réalisant un design optique et mécanqiue pour une adaptation sur le télescope de la HEIG-VD.
% Un programme a été constitué afin de traiter les données de l'image et de quantifier la perturbation. 
% Les résultats ...

% English
This thesis, in addition to a parallel thesis, was carried out to characterize the IRSOL astronomical observation site (Ticino).
The aim is to quantify the atmospheric disturbance at the observation site. \newline
The tool selected was a system called DIMM (Differential image motion monitor).
It principle is simple, an image is masked in two to produce a differential measurement.\newline
Differential measurement cancels out any external disturbances that may occur around the telescope (e.g. movements and vibrations).
\bigbreak
Atmospheric disturbance will induce movements in this image, and the seeing will be qualified by the standard deviation of the measurement.
The result can be expressed as the standard deviation of the beam incidence angle ($\sigma$) or as the corresponding Fried parameter ($r_0$).
\bigbreak
The project was completed with an optical and mechanical design for adaptation to the \Gls{heig-vd} telescope.
A program was created to process the image data and quantify the disturbance. \newline
The results ...