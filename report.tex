\documentclass[
    iai, % Saisir le nom de l'institut rattaché
    mi, % Saisir le nom de l'orientation
    %confidential, % Décommentez si le travail est confidentiel
]{heig-tb}

\usepackage[nooldvoltagedirection,european,americaninductors]{circuitikz}
\usepackage[final]{pdfpages}
\usepackage{hyperref}
\DeclareMathSymbol{*}{\mathbin}{symbols}{"01}   % Replace "*" with middle point

\signature{Paillard.svg} % Remplacer par votre propre signature vectorielle.

%\makenomenclature
\makenoidxglossaries
\makeindex

\addbibresource{bibliography.bib}

\newacronym{ao}{AO}{Adaptative Optics}

\newglossaryentry{heig-vd}{
    name=HEIG-VD,
    description={Haute École d'Ingénierie et de Gestion du canton de Vaud}
}
\newglossaryentry{hes-so}{
    name=HES-SO,
    description={Haute École Supérieure de Suisse Occidentale}
}
\newglossaryentry{IRSOL}{
    name=IRSOL,
    description={Istituto ricerche solari Aldo e Cele Daccò}
}

% Auteur du document (étudiant-e) en projet de Bachelor
\author{Ervan Paillard}

% Activer l'option pour l'accord du féminin dans le texte
\genre{male}

% Titre de votre travail de Bachelor
\title{Optical turbulence analyzer for the IRSOL telescope}

% Le sous titre est optionnel
\subtitle{Bachelor thesis}

% Nom du professeur responsable
\teacher {Prof. L.Jolissaint de Sepibus (HEIG-VD)}

% Mettre à jour avec la date de rendu du travail
\date{\today}

% Numéro de TB
\thesis{11112}



\surroundwithmdframed{minted}

%% Début du document
\begin{document}
\selectlanguage{English}
\maketitle
\frontmatter
\clearemptydoublepage
%\newpage

%% Requis par les dispositions générales des travaux de Bachelor
\pagenumbering{Roman}


\preamble
\includepdf[page=-,scale=0.9]{assets/figures/CDC/CahierDesCharges.pdf}
\Authentication

%% Résumé / Résumé publiable / Version abrégée
\begin{abstract}
    % Francais
% Cette thèse, en complément à une thèse réalisée en parrallèle, a été réalisée dans le but de caractériser le site 
% d'observation astronomique de IRSOL (Tessin).
% L'objectif est de quantifier la perturbations atmosphérique du site d'observation.
% L'outil qui a été sélectionné est un système appelé DIMM (Differential image motion monitor). 
% Le principe est simple. Il suffit de masquer une image en deux pour en faire une mesure différentielle. 
% La mesure différentielle permet d'annuler les perturbations extérieures qui pourraient apparaîtres autour du télescope 
% (par exemple les mouvements et vibrations)
% La perturbation atmosphérique induira des mouvements de cette image et le seeing se qualifiera grace à l'écart type de la mesure.
% Le résultat pourra être exprimé avec l'écart type de l'angle d'incidence des faisceau ($\sigma$) ou par le paramètre de Fried ($r_0$) correspondant.
% Le projet a été mené a bien en réalisant un design optique et mécanqiue pour une adaptation sur le télescope de la HEIG-VD.
% Un programme a été constitué afin de traiter les données de l'image et de quantifier la perturbation. 
% Les résultats ...

% English
This thesis, in addition to a parallel thesis, was carried out to characterize the IRSOL astronomical observation site (Ticino).
The aim is to quantify the atmospheric disturbance at the observation site. \newline
The tool selected was a system called DIMM (Differential image motion monitor).
It principle is simple, an image is masked in two to produce a differential measurement.\newline
Differential measurement cancels out any external disturbances that may occur around the telescope (e.g. movements and vibrations).
\bigbreak
Atmospheric disturbance will induce movements in this image, and the seeing will be qualified by the standard deviation of the measurement.
The result can be expressed as the standard deviation of the beam incidence angle ($\sigma$) or as the corresponding Fried parameter ($r_0$).
\bigbreak
The project was completed with an optical and mechanical design for adaptation to the \Gls{heig-vd} telescope.
A program was created to process the image data and quantify the disturbance. \newline
The results ...
\end{abstract}

%% Sommaire et tables
\clearemptydoublepage
%\newpage

{
    \tableofcontents
    \let\cleardoublepage\clearpage
    \listoffigures
}

\printnomenclature
\clearemptydoublepage
%\newpage

\printglossaries
\clearemptydoublepage
%\newpage

\pagenumbering{arabic}

%% Contenu
\mainmatter
\chapter{Introduction}
%%
This chapter introduces the context and generalities relative to the work.



%====================================================================================================
% Context
%====================================================================================================
\section{Context}
This bachelor's thesis was carried out to support the research of the institute \Gls{IRSOL}. \newline
<<\Gls{IRSOL} is a research institute dependent on a foundation of the same name and affiliated to Università della Svizzera italiana.>>$^1$
The institute is based above Locarno in Ticino. \newline
Their research is mainly based on the measurement and characterization of the sun, and specializes in :
<<
\begin{enumerate}
    \item observational spectropolarimetry and instrument development,
    \item theoretical modeling of the generation and transfer of polarized radiation,
    \item numerical simulation of the solar atmosphere and numerical radiative transfer.
\end{enumerate}>>\footnote{\cite{IRSOL_research} IRSOL, 2023. RESEARCH ACTIVITY, Research Activity at IRSOL. [Online]. [Accessed on 21.07.2023]. Available at: https://www.irsol.usi.ch/research/research-activity/}

The problem is that the institute is looking to expand its equipment in order to develop their research.
However, it is necessary to know the limitations of the measuring instruments, as well as the quality of the observation site.
Their first development objective would be to integrate an adaptive optics system \acrfull{ao}.
\newline
In odrer to asses the usefullness of adaptative optics two Bachelor thesis are conducted.
\begin{enumerate}
    \item (Celui de DOM)
    \item This thesis : "Analyseur de la turbulence optique pour le télescope solaire IRSOL" (Optical turbulence analyser for \Gls{IRSOL}'s solar telescope)
\end{enumerate}
The 1\textsuperscript{st} one is meant to measure the optical imperfections of the telescope itself and maybe of other optical equipment.
The 2\textsuperscript{nd} one aims to analyse optical turbulence on site.

Once these two projects are completed, IRSOL will be able to determine whether the acquisition of an \acrfull{ao} makes sense
and they will also be able to quantify the limiting element to their research (the observation site or the telescope).



%====================================================================================================
% Issue at hand
%====================================================================================================
\section{Issue at hand}
In a discussion with the institute, it was said: "Seeing evolves over the course of the day.
When the sun rises, the ground warms up and the heat rises. This is when seeing is at its worst.
But when the sun goes down, there's a moment when seeing is almost non-existent because the air currents freeze before going down again."
\bigbreak
The issue at hand is that the quality of the observation site is known but cannot be quantified.
The aim of this study is to quantify the quality of their observation site.
The instrument used will be the DIMM \textbf{\textcolor{red}{RAJOUTER GLOSSAIRE DIMM}} (Differential image motion monitor).

A DIMM \textbf{\textcolor{red}{RAJOUTER GLOSSAIRE DIMM}} works as follows (simplified version) :
\begin{enumerate}
    \item Capture an image masked by two small holes.
    \item Tracking the movement of these images.
    \item Measurement of the standard deviation of this movement.
    \item Conversion of previous result into turbulent layer thickness.
\end{enumerate}
The aim will be to play with the exposure time to determine the location of the image center, but also to freeze the turbulence.
\bigbreak
The issue at hand is that the the teslescope has some unkown optical imperfecitons. This study aims to measure them.

The short methodology explanation is as follows:
\begin{enumerate}
    \item Sets of 2 pictures are taken (1 focused, 1 unfocused (in a known way))
    \item The best sets are kept and stacked.
    \item The stack is run throug a code which output will then be interpreted to find the imperfections.
\end{enumerate}

If the exposure time is short enough, turbulences will eventually be fixed into place, resulting in a sharp image.\footnote{\cite{Law_lucky} Law, 2006. Lucky imaging: High angular resolution imaging in the visible from the ground. in: Astronomy \& Astrophysics [online], February 2006, N°446, pp. 739-745. ISSN
1950-6295. [Accessed on 24.02.2023]. Available at: https://www.aanda.org/articles/aa/abs/2006/05/} Furthermore, if the time between a set of 2 pictures is short enough, the perturbation will not have moved which means a set has the same perturbations for both pictures.

In a perfect world, the exposures time is infinitely small and the time inbetween two frames inexistent. This will be further discussed.
The process of stacking images is called lucky imaging. The theory behind it it that the mean turbulence is null\footnote{\cite{JOLISSAINT_master} JOLISSAINT, Laurent, 1996. XXXXX. Genève : XXXX. Master's thesis}, thus the result should be a picture free of aberations caused by the atmosphere.

Alternatively, long exposure times will also cancel the effect of atmospheric turbulence.

The long explanation is the subject of this document.



\chapter{Theory}
%====================================================================================================
% Fundamentals
%====================================================================================================
\section{Fundamentals}
This part goes over the fundamentals of optics necessary to understand future developments.
%====================================================================================================
% Seeing
%====================================================================================================
\subsection{Seeing}
\textbf{What's the seeing ?} : \newline In atmospheric optics, the "seeing" refers to the degree of blurring or distortion of astronomical objects
caused by Earth's atmosphere. The Earth's atmosphere is not homogeneous. It has various layers of air with different densities and temperatures.
These different layers cause the light passing through it to be refracted in different ways. This result causes a constant change of blurring,
distortion and glittering. Especially when looking an object low on the horizon. In our case, at IRSOL, the seeing is 4' of arc. We can go up
to 1' arc in the best case\newline
In general, seeing is determined by the angle of an object seen through the atmosphere, which is affected by factors such as temperature and wind speed.
The result of the seeing measure is the inverse of the "Fried parameter", which describes the size of the atmospheric cells that cause the blurring.
When the Fried parameter is small, we could say that the seeing has good conditions. That means the image will have minimal blurring and distortion.
While poor seeing conditions have a larger Fried parameter, resulting in significant blurring and distortion of atmospherical objects.
%(Paramètre de Fried => Inverse)
\bigbreak
\textbf{Where are the turbulences} : \newline
First, we need to know how much type of turbulence we could see appearing and where they could be appear.\newline
There are 4 types of turbulence that we could see on the atmosphere. Each type is listed below :
\begin{itemize}
    \item Dome turbulence : \newline Appear when the air as not the same temperature between inside and outside the dome. This turbulence corresponds to
          the mirror turbulence.
    \item Surface trubulence : \newline Appear between the first 10 to 100 meters. Its appearance is due to the cooling by convection of the ground
          heated by the sun. Its minimum is just after the sunrise and its minimum when the sun sets. The evolution of this turbulence is an increasing
          evolution before the zenith and a decreasing evolution after it.\newline
          To counter this turbulence, a site selection is recommanded. For minimize it, the telescope could be place on a tower, away from any surface.
    \item Medium altitude turbulence : \newline Appear between 1.5 to 6 kilometers. Its appearance is due to atmospheric streams and in the thermal
          instabilities of the atmosphere. This turbulence is made up of multitudes of fine turbulent layers of varying densities and of a few hundred
          meters. For minimize it, a measure on the site should be carried out before.
    \item Tropopause and stratospheric turbulence : \newline Could appear between 6 and 20 kilometers. It reach a minimum t 6km and maximum on the
          tropopause between 10 and 20km. Its appearance is due to the strong winds shearing the atmosphere. This tubulence decrease after reach its maximum
          until it disappears after 25 - 30km.
\end{itemize}

%====================================================================================================
% DIMM
%====================================================================================================
\subsection{Differential image motion monitor (DIMM)}
\textbf{What's a DIMM ?} : \newline A differential image motion monitor is an instrument used in atmospheric optics to measure the amount of turbulence
in the Earth's atmosphere. It analyze the movement of stars as their light passes through the atmosphere, wich causes the stars to twinkle and blur.
\newline
The DIMM system consists of mask and split the image in two. After that, each image is recorded with a camera. On each part of this camera, the image
has a different seeing. Theses differences are used to determine the amount of distortion caused by the atmosphere.\newline
If we made a differential measure of theses images, we will get the atmospheric turbulence, wich is expressed as the seeing or the size of the
blurred image of a point source. \newline
The DIMM technique is widely used in astronomical observations, as it provides an objective and quantitative measure of the atmospheric turbulence,
which can affect the resolution and sensitivity of telescopes. It is also used in adaptive optics systems, where it is used to measure the atmospheric
turbulence in real-time, and to correct for the distortions using deformable mirrors.

\bigbreak
\textbf{Structures} : \newline the DIMM system is a relatively simple and robust instrument, which provides an accurate and objective measure of the atmospheric turbulence.
The elements that constitute a Differential Image Motion Monitor (DIMM) are listed below and viewable on the figure
\ref{fig:DIMM_Schematic} :
\begin{enumerate}
    \item Optical path : To create two images of our object, we need something wich will split our object in two. To realize this, a beam splitter could
          should be the best solution. It is also possible to separate the object before it enters the telescope by installing a "mask" before its first
          eyepiece.
    \item Camera : Which records the images of the star. The cameras are usually high-speed, sensitive detectors, such as CCDs or EMCCDs.
    \item Image processing software : The images obtained by the cameras are analyzed by software that measures the distortion caused by the atmosphere.
          The software calculates the difference in the distortion measurements obtained by the two telescopes, which allows for the determination of the
          atmospheric turbulence.
    \item Control and data acquisition system : The DIMM system is usually controlled by a computer, which also acquires and stores the data obtained by
          the cameras. The data can then be used to calculate the atmospheric turbulence, and to analyze the seeing conditions.
\end{enumerate}
\begin{figure}[H]
    \centering
    \includegraphics{assets/figures/Theory/DIMM_Schematic.jpg}
    \caption{Schematic principe of DIMM}
    \label{fig:DIMM_Schematic}
\end{figure}

%====================================================================================================
% Equation
%====================================================================================================
\section{Usual mathematical functions}
The basic mathematical functions used and quoted in the theoretical part are found in this section.
\subsection{Nyquist frequency}
\begin{equation}\label{eq:Nyquist}
      n_1*\sin(\theta_1)=n_2*\sin(\theta_2)
\end{equation}
\subsection{Refraction with Snell-Descartes}
\begin{equation}\label{eq:Snell}
      n_1*\sin(\theta_1)=n_2*\sin(\theta_2)
\end{equation}
\subsection{Trigonometry}
\begin{equation}\label{eq:Trigo_sin}
      sin(\alpha) = \frac{opposite}{Hypotenuse}
\end{equation}
\begin{equation}\label{eq:Trigo_cos}
      cos(\alpha) = \frac{Adjacent}{Hypotenuse}
\end{equation}
\begin{equation}\label{eq:Trigo_tan}
      tan(\alpha) = \frac{opposite}{Adjacent}
\end{equation}
\subsection{Equation of the affine line}
\begin{equation}\label{eq:DoiteAffine}
      Y = a*X+b
\end{equation}
\subsection{law of conjugate foci}
\begin{equation}\label{eq:Conjugate}
      \frac{1}{f}=\frac{1}{p_i}+\frac{1}{p_o}
\end{equation}


\chapter{State of the art}
The \Gls{DIMM} concept is well known in the world of astrophysics, and has already been implemented in several forms.
However, most of the \Gls{DIMM}s produced refer to article "The ESO differential image motion monitor" |3|.
Mr Sarazin and Mr Roddier have produced a seeing monitor for ESO and describe it in this article.

\section{Principle}
The principle is the same as that described in section \ref{sec:Theo_DIMM}, with the light beam passing through a turbulent layer
before reaching the telescope. This beam is then masked, deflected by a prism and finally focused on a screen to produce
2 distinct images. In the next section, we'll be working out the seeing measurement algorithm. All the calculations involved
in determining seeing are listed and explained in ESO's article on DIMM |3|.
\section{Realization}
When it comes to \Gls{DIMM} realizations, there are several variants :
\begin{itemize}
    \item Mask installed in front of the telescope pupil.
    \item Mask and prisms installed in front of the telescope pupil.
    \item Mask and prisms installed after telescope (Mask on exit pupil).
    \item Prism installed only for one of the masked beams.
    \item Prism installed for 2 masked beams.
\end{itemize}
All these solutions have achieved results, but the methods of image processing and acquisition are never the same.
\newpage
\section{Alternative methods}
There are several ways of measuring seeing other than \Gls{DIMM}. Some of these methods are listed below :
\begin{itemize}
    \item \textbf{Lucky imaging} : Lucky Imaging takes advantage of the fact that atmospheric turbulence changes
          rapidly over time. The technique involves capturing a large series of images very quickly,
          usually in a few milliseconds. This high frame rate enables us to freeze moments of less severe atmospheric
          turbulence and to reject images captured during moments of high turbulence. Once a clear image of the object
          has been recovered, instant seeing can be determined by taking further successive images, using the image
          obtained by Lucky imaging as a reference.
    \item \textbf{LSIM (Least-Squares Image Motion)} : Unlike Lucky Imaging, which selects lucky images, the LSIM
          method uses all the images captured in the sequence to calculate a single global turbulence displacement,
          which it then corrects to improve the quality of the final image. As with the LSIM method, the seeing can then
          be determined using the reference image found with this method.
    \item \textbf{Scintillation Analysis} : This method is more similar to \Gls{DIMM}. However, the sensor used is a photometer.
          This device is used to record the star's brightness at regular time intervals. This allows us to capture variations
          in brightness due to flicker. The flicker's characteristics are compared with theoretical atmospheric models
          to obtain the current seeing.
\end{itemize}

\chapter{Optical design}
\section{Principe}
The principle of the optical design to be realized is schematized on the figure \ref{fig:OpticalPrincipe}.\bigbreak
The output beam of the telescope must be made parallel before the system can be installed. Once the beam is parallel,
a mask composed of 2 equal openings will limit it. This mask correspond to the entry of the system.\newline
The beam will enter to a prism which will deflect the beams to a lens.
The lens will intersept the 2 beams and allow to focus them on a sensor (BW camera).
\begin{figure}[H]
    \centering
    \includegraphics[scale=0.7]{assets/figures/Optical Design/Schéma optique.png}
    \caption{Optical schematic principe}
    \label{fig:OpticalPrincipe}
\end{figure}
\newpage
%====================================================================================================
%====================================================================================================
% Drawing rays with matlab
%====================================================================================================
%====================================================================================================
\section{Drawing rays with matlab}
This part will correspond to the mathematical research to draw the rays on a matlab program. To simplify the program,
the chosen method is to draw lines through 2 points. The calculated points are the following:
\begin{enumerate}
    \item Prism entry = ($X_{PE1,2,3,4},Y_{PE1,2,3,4}$)
    \item Prism exit = ($X_{PO1,2,3,4},Y_{PO1,2,3,4}$)
    \item Lens entry/exit = ($X_{L1,2,3,4},Y_{L1,2,3,4}$)
    \item Screen entry = ($X_{S12,34},Y_{S12,34}$)
\end{enumerate}
Known data :
\begin{itemize}
    \item Prism angle = $\alpha$ [$\deg$] (converted into radian for the equations)
    \item Deflection of the incident beam = $i$ [" arc] (converted into radian for the equations)
    \item focal lens = $F$ [mm]
    \item Hole diameter = $d$ [mm]
    \item Distance betweeen two holes = $e$ [mm]
    \item Distance between the prism and the lens = $E$ [mm]
    \item Refractive index of the prism = $n_p$ [-]
    \item Prism width = $t_c$ [mm]
    \item Prism height = $h$ [mm]
\end{itemize}
Needed data :
\begin{itemize}
    \item Screen height = $H$ [mm]
    \item System length = $L$ [mm]
\end{itemize}
\newpage
%====================================================================================================
% Prism
%====================================================================================================
\subsection{Prsim}
\begin{figure}[H]
    \centering
    \includegraphics[scale=0.8]{assets/figures/Optical Design/Schéma prisme.png}
    \caption{Schematic of drawing rays at the input of the prism}
    \label{fig:Prism_Schematic}
\end{figure}
First, set the X and Y input coordinates.
\begin{equation}
    X_{PE1} = X_{PE2} = X_{PE3} = X_{PE4}
\end{equation}
\begin{equation}
    Y_{PE1} = \frac{e + d}{2} \newline
    Y_{PE2} = \frac{e - d}{2} \newline
    Y_{PE3} = \frac{-e - d}{2} \newline
    Y_{PE4} = \frac{-e + d}{2}
\end{equation}
Using Snell's law and the relationships between the angles, the exit angle can be determined.
\begin{equation}
    r = \arcsin(\frac{\sin(i)}{n_p})
\end{equation}

\begin{equation}
    \beta = \arcsin(n_p*\sin(\alpha-\arcsin(\frac{\sin(i)}{n_p})))-\alpha
\end{equation}
To find the output values, the affine line function (\ref{eq:DoiteAffine}) is used.
This requires knowledge of the slopes and offsets of the straight lines (beam + prism output face).
\begin{equation}
    m_1 = \frac{-1}{\tan(\frac{\pi}{2}-r)}\newline
    m_2 = \frac{-1}{\tan(\alpha)}
\end{equation}
\begin{equation}
    m_3 = \frac{-1}{\tan(\frac{\pi}{2}-r)}\newline
    m_4 = \frac{1}{\tan(\alpha)}
\end{equation}
\begin{equation}
    b_1 = Y_{PE1}-X_{PE1}*m_1\newline
    b_2 = Y_{PE2}-X_{PE2}*m_2\newline
    b_{1s} = \frac{h}{2}-t_e*m_2
\end{equation}
\begin{equation}
    b_3 = Y_{PE3}-X_{PE3}*m_3\newline
    b_4 = Y_{PE4}-X_{PE4}*m_4\newline
    b_{2s} = \frac{-h}{2}-t_e*m_4
\end{equation}
Now that the line parameters are known, the X and Y output points of each beam can be calculated.
\begin{equation}
    X_{PO1} = \frac{b_{1s}-b_1}{m_1-m_2}\newline
    Y_{PO1} = X_{PE1}*m_1+b_{1s}
\end{equation}
\begin{equation}
    X_{PO2} = \frac{b_{1s}-b_2}{m_1-m_2}\newline
    Y_{PO2} = X_{PE2}*m_2+b_{1s}
\end{equation}
\begin{equation}
    X_{PO3} = \frac{b_{2s}-b_1}{m_1-m_2}\newline
    Y_{PO3} = X_{PE1}*m_1+b_{1s}
\end{equation}
\begin{equation}
    X_{PO2} = \frac{b_{1s}-b_2}{m_1-m_2}\newline
    Y_{PO2} = X_{PE2}*m_2+b_{1s}
\end{equation}
Once the X,Y coordinates of the prism's inputs and outputs are known, it's easy to plot them in matlab (Appendix \ref{App:Matlab}).
%====================================================================================================
% Lens
%====================================================================================================
\subsection{Lens}
\begin{figure}[H]
    \centering
    \includegraphics{assets/figures/Optical Design/Schéma sortie prisme.png}
    \caption{Schematic of drawing rays at the output of the prism}
    \label{fig:PrismLens_Schematic}
\end{figure}
First, let's say :
\begin{equation}
    X_L = E
\end{equation}
With the equation of the affine line (\ref{eq:DoiteAffine}),
where "a" can be determined using trigonometry (\ref{eq:Trigo_tan}),
The offset can be determined using the points previously obtained.
\begin{equation}
    a = \frac{-1}{tan(90-\beta)}
\end{equation}
\begin{equation}
    Y_{out} = a*X_{out} + b ==> b = Y_{out} - a*X_{out}
\end{equation}
We can now pose the equation that determines the $Y_L$ coordinate
\begin{equation}
    Y_{L} = a*X_{L} + b
\end{equation}
Once the X,Y coordinates of the lens's inputs are known, and with the prism's output coordinates, it's easy to plot them in matlab (Appendix \ref{App:Matlab}).
\subsection{Screen}
\begin{figure}[H]
    \centering
    \includegraphics{assets/figures/Optical Design/Schéma Lentille.png}
    \caption{Schematic of drawing rays at the output of the lens}
    \label{fig:Lens_Schematic}
\end{figure}
To begin with, the X coordinate can be set as it corresponds to the focal
point of the lens. Note that this equation is only valid if the lens is approximated to a thin lens.
\begin{equation}
    X = E+F
\end{equation}
Then, thanks to trigonometry (\ref{eq:Trigo_tan}), it is easy to find the coordinates Y
\begin{equation}
    Y = \pm F*tan(\beta)
\end{equation}
Once the X,Y coordinates of the screen's inputs are known, and with the lens's output coordinates, it's easy to plot them in matlab (Appendix \ref{App:Matlab}).
%====================================================================================================
%====================================================================================================
% Matrice de conjuguaison avec matlab
%====================================================================================================
%====================================================================================================
\newpage
\section{Conjugation matrix with matlab}
blblalbalblabl
%====================================================================================================
%====================================================================================================
% Camera specifications
%====================================================================================================
%====================================================================================================
\section{Camera specifications}
In order to determine which sensor will have the right specifications for this application,
several elements will need to be calculated :
\begin{itemize}
    \item Pixel size to ensure an image with sufficient information
    \item Screen size
    \item The size the image will occupy on the sensor
\end{itemize}
\bigbreak
To begin with, the mask must be positioned at the telescope entrance. To do this, the relationship between
the holes and their spacing is used (:::). A virtual mask is placed over the telescope entrance,
with a 20mm margin taken on the sides (10mm each). Using these elements, the diameter of the holes on the
telescope can be determined by the equation \ref{eq:DMT} with $D_{MT}$ = diameter of a hole on the telescope entrance,
$D$ = diameter of the telescope, $margin$ = 20mm, $ratio$ = ratio between the holes and their spacing.
\begin{equation}\label{eq:DMT}
    D_{MT} = \frac{(D-margin)}{(ratio + 1)} = \frac{(305-20)}{5+1} = 47.5 mm
\end{equation}
Now we need to know the transverse magnification ($G_t$) between the holes in the mask and the virtual holes in the telescope.
This magnification will later be used to determine the available FOV.
\begin{equation}
    G_t = \frac{D_{holes}}{D_{MT}} = \frac{1}{47.5} = 0.0211
\end{equation}
\textbf{\textcolor{red}{Truc du sigma}}
\newline
Once the $\sigma_T$ is known, the fov as seen through the telescope can be determined. In equation \ref{eq:FOV_T}, $D_S$ is the diameter
of the sunspots to be measured (20").
\begin{equation}\label{eq:FOV_T}
    FOV_T = 6 * \sigma_T + D_S
\end{equation}
Once the FOV seen by the telescope is known, this parameter must be transferred to the mask.
\begin{equation}
    FOV_M = FOV_T * G_t^{-1}
\end{equation}
This FOV will be converted into a radiant and can then be transferred to the CMOS sensor
using the lens focal length ($F$) determined in section \ref{sec:Opti_Lens}.
\begin{equation}
    FOV_S = FOV_M * F
\end{equation}
Finally, the three elements listed at the beginning of this section can be calculated.
\begin{itemize}
    \item ballbalblalblalba
          \begin{equation}
              pixel_{max} = \frac{\lambda*F}{2*D_{holes}}
          \end{equation}
    \item blablalballbaa
\end{itemize}

%====================================================================================================
%====================================================================================================
% Lens selection
%====================================================================================================
%====================================================================================================
\section{Lens selection}\label{sec:Opti_Lens}
balblalbalblalab
%====================================================================================================
%====================================================================================================
% Prism selection
%====================================================================================================
%====================================================================================================
\section{Prism selection}
blalbalblalb
%====================================================================================================
%====================================================================================================
% Limitation
%====================================================================================================
%====================================================================================================
\section{Limitations of the system}\label{sec:Opti_Limit}
Parler du ratio du masque etc
Parler du filtre solaire pour le télescope

\chapter{Mechanical design}
%====================================================================================================
% En-tête
%====================================================================================================
\begin{figure}[H]
    \centering
    \includegraphics[scale=0.65]{assets/figures/Mechanical Design/Système.png}
    \caption{Representation of the Mechanical design}
    \label{fig:Mec_Blobal}
\end{figure}
\newpage
%====================================================================================================
% Preliminary choices and limitations
%====================================================================================================
\section{Preliminary choices and limitations}
\subsection{Objectives}
First, the mechanical system need to be attached to the output of the laboratory telescope. To achieve this, an adaptor piece is
required (\ref{sec:adaptator}). Then, the masks will need to be attached to a support (\ref{sec:mask}). A second support will have
to be made to accommodate the prisms (\ref{sec:prisms}). Finally, the lens must be installed (\ref{sec:lens}) and the camera placed
just behind it (\ref{sec:Camera}). \newline
The system must be compact, adjustable and adaptable if modifications are required at a later date.
\subsection{Limitations}
To make this system a reality, a simple and common fastening system had to be integrated. The most simple system used in optical design 
is the cage system. Two of the world's leading optomechanics manufacturers offer it (Thorlabs and Edmund Optics). The significant difference 
between these 2 manufacturers is the spacing between the steel bars. This difference is shown in the capter \ref{sec:lens}. \newline
To integrate the telescope's occular correctly, the Edmund optics system is required. This is because the distance between the steel bars 
is smaller on the thorlabs system, and the distance between the occular and the bore for the bar would be far too small ($\approx$ 0.05mm).
\bigbreak
The difference between the Thorlabs system and the Edmund Optics system is the centering diameters for the steel bar holes who 
can be seen on the figure \ref{fig:thorlabs_Edmund} ($\emptyset\ 38\ mm$ for Thorlabs (blue) and $\emptyset\ 57\ mm$ for Edmund optics (red))
\begin{figure}[H]
    \centering
    \includegraphics[scale=1]{assets/figures/Mechanical Design/Thorlabs_Edmund.png}
    \caption{Differences between thorlabs and Edmund systems}
    \label{fig:thorlabs_Edmund}
\end{figure}
%====================================================================================================
% Telescope mounting system
%====================================================================================================
\section{Telescope mounting system}\label{sec:adaptator}
The telescope-mounting system, as its name suggests, will be the connecting piece between the telescope output and the DIMM. It will 
also integrate the telescope's occular. The choice of integrating the telescope's ocular with this part was made to improve the system's 
stability and alignment. The system will be screwed onto the telescope, so it won't be held in place by simple screws attached to 
the part. \newline
This part will maintain all the DIMM. So it need to be strong and more massive than the other parts. However, it also needs to be as 
light as possible so as not to weigh too much on the telescope. Material was therefore removed from the middle of the piece to reduce 
its weight.
\bigbreak
Figure \ref{fig:Mec_Adapter1} (Adapter drawing) shows :
\begin{enumerate}
    \item The hole for the ocular ($\emptyset\ 33.55\ mm$) and the thread for its attachment ($MF\ 40\ P=1$). Fixing is simply done 
    with a threaded ring drilled through the center.
    \item The thread for attaching the system to the telescope ($\emptyset\ 50.75\ mm\ TPI\ 24$)
    \item Hole for steel bars and the screws used to secure them ($\emptyset\ 6.02$)
\end{enumerate}
\begin{figure}[H]
    \centering
    \includegraphics[scale=0.6]{assets/figures/Mechanical Design/Dessin_Part1_adapter.png}
    \caption{Representation of the adaptator : ocular and bar fixation}
    \label{fig:Mec_Adapter1}
\end{figure}
This part will be made of aluminum to reduce its weight. The use of aluminum has no effect on the rigidity of the system. \newline
However, the ocular attachment ring will be made of steel to limit the coefficient of friction when screwing it on 
(screwing an aluminum part with another of this material will have a grip effect between the two parts).
\bigbreak
The complete drawing can be found in the appendix \ref{App:MEP}
\newpage
%====================================================================================================
% Mask mounting
%====================================================================================================
\section{Mask and mounting}\label{sec:mask}
\subsection{Mask}
As explained in section \ref{sec:Opti_Limit}, the mask will have 2 holes 1mm in diameter, with their centers 5mm apart. 
A 0.5mm-thick plate was designed for this purpose. This plate includes the 2 holes and is shaped to fit more easily into its support.
This part is shown in figure \ref{fig:Mec_Mask}
\begin{figure}[H]
    \centering
    \includegraphics[scale=0.55]{assets/figures/Mechanical Design/Dessin_Mask.png}
    \caption{Drawing of the mask}
    \label{fig:Mec_Mask}
\end{figure}
\subsection{Mounting}
For the support, 2 pieces are required. The first will contain the "rail" where the mask will be inserted (Figure \ref{fig:Mec_Mask_Sup_Rail} left),
 and the other will hold it in place (Figure \ref{fig:Mec_Mask_Sup_Rail} right).
\begin{figure}[H]
    \centering
    \includegraphics[scale=0.85]{assets/figures/Mechanical Design/Support_masque_rail.png}
    \caption{Supports for the mask}
    \label{fig:Mec_Mask_Sup_Rail}
\end{figure}
Figure \ref{fig:Mec_Mask_Sup_Rail} (left) shows the mask positioning rail, the tapped holes for attaching the second part and the holes 
for the steel bars. \newline
Figure \ref{fig:Mec_Mask_Sup_Rail} (right) shows the second piece who is designed for easy mask removal. This part also features the opening 
for inserting the mask, as well as the screw holes for securing the two parts together.\bigbreak
This part will be made of aluminum to reduce its weight. The use of aluminum has no effect on the rigidity of the system.
\bigbreak
The complete drawing can be found in the appendix \ref{App:MEP}
%====================================================================================================
% Prisms and mounting
%====================================================================================================
\section{Prisms and mounting}\label{sec:prisms}
\subsection{Prisms}
As explained in section \ref{sec:Opti_Prism}, the output beams from the prism must have an angle of incidence of 1°. 
This corresponds to an angle of 1.93° on the prism. \newline
Several requests for bids were made, but none were within the budget allocated under the TB. The aim was to have a square-shaped 
prism (\ref{fig:Prism_square} left) to reduce the size of the system and make it easier to assemble. However, this solution could not be retained and the 
use of circular prisms (\ref{fig:Prism_square} right) was required. This solution greatly reduces the budget, but the part that will hold them will be much more complex.
\begin{figure}[H]
    \centering
    \includegraphics[scale=0.5]{assets/figures/Mechanical Design/prisme_voulu.png}
    \caption{Needed prism / Prism choosed}
    \label{fig:Prism_square}
\end{figure}
\subsection{Mounting}
To attach the 2 prisms, we had to create a custom-made part. To achieve this, several elements were critical 
like direction, position and orientation of the prism.
\bigbreak
For these purpose, a maintenance part has been created (Figure \ref{fig:Prism_Support1}). This part will allow the 2 prisms to be inserted, while letting 
the beam from the mask pass through. \newline
There are 2 small notches on the back of the part. These notches are there to ensure correct mounting of the prism's 
holding parts (Figure \ref{fig:Prism_Support2}).\newline
This part also has holes and threads for fixing the assembly to the steel bars, and for attaching the part that 
will hold the prisms in position. 
\begin{figure}[H]
    \centering
    \includegraphics[scale=0.55]{assets/figures/Mechanical Design/MaintientPrism1.png}
    \caption{Support for the prisms}
    \label{fig:Prism_Support1}
\end{figure}
The part shown in figure \ref{fig:Prism_Support2} was created to fix and hold the prisms correctly. This part fits perfectly into the support 
bore and will be positioned with the notches. \newline 
This part is made with the same angle of inclination as the prisms on one side, so as to be able to fix them perfectly. 
If the prism is not correctly positioned, the part will protrude more strongly from its support. If the prism is in its 
deepest position, it will be correctly mounted and the clamp can be installed.\newline
During assembly, this part requires particular attention. They must be handled with care to avoid damaging the prisms.
\begin{figure}[H]
    \centering
    \includegraphics[scale=0.7]{assets/figures/Mechanical Design/MaintientPrism2.png}
    \caption{Prism wedge}
    \label{fig:Prism_Support2}
\end{figure}
The last piece will be used to hold the assembly in position (Figure \ref{fig:Prism_Support3}). It rests on the prism wedges 
and is attached to the support piece (Figure \ref{fig:Prism_Support1}). \newline
During assembly, this part requires particular attention. When tightening the parts, you'll need to tighten each screw a 
little at a time, so that the part remains as parallel as possible to the support.
\begin{figure}[H]
    \centering
    \includegraphics[scale=0.8]{assets/figures/Mechanical Design/MaintientPrism3.png}
    \caption{Clamp for prism support}
    \label{fig:Prism_Support3}
\end{figure}
All these parts will be made of aluminum to reduce its weight. The use of aluminum has no effect on the rigidity of the system.
\bigbreak
The complete drawing can be found in the appendix \ref{App:MEP} for the support and \ref{App:Edmund_MEP} for prisms
%====================================================================================================
% Lens and mounting
%====================================================================================================
\section{Lens and mounting}\label{sec:lens}
\subsection{Lens}
As explained in section \ref{sec:Opti_Limit}, the lens need to be achromatic with a spectrum from 425nm to 675nm 
and a focal lenght of 50mm. For mechanical reasons (size of supports), a 25mm diameter lens had to be integrated. 
\begin{figure}[H]
    \centering
    \includegraphics[scale=0.65]{assets/figures/Mechanical Design/Lentille.png}
    \caption{Achromatic lens F = 50mm (425nm - 675nm)}
    \label{fig:Lentille}
\end{figure}
\subsection{Mounting}
This part of the mechanics is much simpler because everything has been ordered from the Edmund optics website. 
The mechanics used to hold the lens in place are a support that attaches to the steel bars, and a cage for 
the lens that attaches directly to the support.
\begin{figure}[H]
    \centering
    \includegraphics[scale=0.55]{assets/figures/Mechanical Design/Support_Lentille.png}
    \caption{Support and cage for the lens}
    \label{fig:Lentille_Support1}
\end{figure}
\bigbreak
The complete drawing can be found in the appendix \ref{App:Edmund_MEP}
\newpage
%====================================================================================================
% Camera mounting
%====================================================================================================
\section{Camera mounting}\label{sec:Camera}
\subsection{Camera}
As explained in section \ref{sec:Opti_Cam}, the chosen camera is Allied Vision's Alvium 1800 U-052m. \newline
It can be mounted on the front panel using either the C-mount thread or four tapped 
holes. It is also possible to attach to these sides with four additional threaded holes.
\begin{figure}[H]
    \centering
    \includegraphics[scale=0.35]{assets/figures/Mechanical Design/Camera.png}
    \caption{Selected camera}
    \label{fig:Camera}
\end{figure}
It was decided to attach the camera via its C-mount thread. After receiving the camera, it was noted that the 
optimum operating temperature for the electronics was 85°C, which causes the casing to heat up considerably. 
It might be necessary to add coolers (or even a fan) at a later date if the tests are not conclusive. 
Other threaded holes could be used to cool the system (cooler attachment).
\subsection{Mounting}
For the camera, a bracket will be designed to hold it in position. The housing has a threading system 
(C-mount) that is very common in the optical world. This is a mounting used for lenses that can be added 
to this type of camera. The part shown in figure \ref{fig:Camera_Support} corresponds to the bracket and comprises 
an external C-mount thread around the shoulder and holes for steel bars.
\break
The camera will be screwed onto the bracket. Its orientation will pose no problem, as the images of our 
stars/sunspots will always be in the sensor area. However, the software will have to be designed to find 
the areas of interest easily and efficiently.
\begin{figure}[H]
    \centering
    \includegraphics[scale=0.75]{assets/figures/Mechanical Design/Support_Camera.png}
    \caption{Support for the camera}
    \label{fig:Camera_Support}
\end{figure}
\bigbreak
The complete drawing of the support can be found in the appendix \ref{App:MEP} and the drawing of the camera 
in the appendix \ref{App:Edmund_MEP}.

\chapter{Software}
%====================================================================================================
% Possibilities and choices
%====================================================================================================
\section{Possibilities and choices}
\subsection{Possibilities}
\subsubsection{measurement method}
Two options are available for creating the software:
\begin{enumerate}
    \item \textbf{Pre-capture of images using Vimba software (\ref{sec:Vimba})} : \newline
          In this case, the images are taken using the software supplied with the camera and then entered into the software develop with python.
          The problem is that the operator carrying out the measurement will have to switch between initialization parameters and parameters
          for measurements on the Vimba software. For this solution, different software modes will also be required
    \item \textbf{Image capture and analysis} :\newline
          The second solution involves automatic measurement and initialization. The operator would simply point the telescope at an
          object and launch the software. The problem with this solution is that it would be more complex to detect a bug. In this case,
          we'd have to come up with solutions to recognize measurement bugs as quickly as possible.
\end{enumerate}
\subsubsection{Initialization}
For initialization, two solutions are also possible :
\begin{enumerate}
    \item The first is to take several images with a exposure time equal to the measurement part. By adding up all the images,
          the turbulence will be averaged. It will then be easy to determine the object's center of gravity and its centroid.
    \item The second option would be to considerably increase the exposure time of the image. This would be the same as the
          previous option, but taking only 1 image would reduce memory usage and increase computing speed.
\end{enumerate}
In these 2 cases, the number of images to be used or the exposure time required to correctly average the image will need to
be determined during the initial tests. If turbulence is not properly averaged, aberrant results may appear.
\subsubsection{Measurement}
bablablalba
\subsection{Choices}
\subsubsection{measurement method}
balbalblaba
\subsubsection{Initialization}
blalbalblalb
\subsubsection{Measurement}
bablablalba
%====================================================================================================
% Vimba software
%====================================================================================================
\newpage
\section{Vimba software}\label{sec:Vimba}
The Vimba software package was created by Allied Vision and is available to all users of Allied Vision cameras.
The software lets you modify camera parameters in real time (Figure \ref{fig:Soft_Vimba}). It is also designed to pre-process
the image before sending it for output. Inputs are also available to control the camera.
\begin{figure}[H]
    \centering
    \includegraphics[scale=0.85]{assets/figures/Software/VimbaInterface.png}
    \caption{Editable image parameters on Vimba}
    \label{fig:Soft_Vimba}
\end{figure}
This application is very useful for adjusting parameters during measurements, but also for initial tests, as several images can
be taken in quick succession using one of the software's functions (With the "Image series options" parameter).
%====================================================================================================
% Python API for Vimba
%====================================================================================================
\newpage
\section{Python API for Vimba}
The Vimba software introduced in section \ref{sec:Vimba} is also supplied with an API for simulating the software on python.
All API data is given in a git directory (\href{https://github.com/alliedvision/VimbaPython}{Github link}) and the user manual
can be found in appendix \ref{App:pythonVimba}.\newline
Please note that if the Vimba program is open on the computer, the API will not be able to communicate with the camera.
\subsection{Camera connection}\label{sec:Soft_API_Connect}
Before using the camera with the API, the code must call the procedure that simulates the Vimba program :
\begin{verbatim}
    with Vimba.get_instance() as vimba:
        cams = vimba.get_all_cameras()
        with cams[0] as cam:
            // Your code
\end{verbatim}
In the code above, the "cam" structure is the element to be called in order to access all the camera parameters.
With this structure, image parameters, setups and access to images can be carried out.
\subsection{Parameter modification}
The API also lets you change camera parameters separately. In the case of the dimm application, the most important
need is to change the exposure time. To do this, once the camera is connected (Code in section \ref{sec:Soft_API_Connect}),
call this function:
\begin{verbatim}
    exposure_time.set(2000) #us
\end{verbatim}
This function lets you change the exposure time (given in microseconds) and can range from 20 to 10000000 (10 seconds).
\subsection{Load and save settings}
\subsubsection{Save settings}
Pre-recorded settings files can be created for the camera. If certain values are modified and you wish to keep them,
you can save the file on your computer. To do this, after connecting to the camera (\ref{sec:Soft_API_Connect}),
simply call up the function :
\begin{verbatim}
    settings_file = 'my_File.xml'
    cam.save_settings(settings_file, PersistType.All)
\end{verbatim}
\subsubsection{Load settings}
To retrieve settings that have already been saved and load them into the camera, use the code below:
\begin{verbatim}
    settings_file = 'my_File.xml'
    cam.load_settings(settings_file, PersistType.All)
\end{verbatim}
\newpage
\subsection{Image capture}
Before taking an image, you need to be able to set a format compatible with the compiler or 
a library that will be used to create the software. To do this, the following code compares 
the camera's available formats with the compatible formats.
\begin{verbatim}
    with Vimba.get_instance() as vimba:
        cams = vimba.get_all_cameras()
        with cams[0] as cam:
            formats = cam.get_pixel_formats()
            opencv_formats = intersect_pixel_formats(formats, OPENCV_PIXEL_FORMATS)
        print(f"Available formats:")
        for i, format in enumerate(formats):
            print(i, format)
        print(f"\nOpencv compatible formats:")
        for i, format in enumerate(opencv_formats):
            print(i, format)
\end{verbatim}
\begin{figure}[H]
    \centering
    \includegraphics[scale=0.85]{assets/figures/Software/Format_Available.png}
    \caption{Compatible format results}
    \label{fig:Soft_Vimba_Format}
\end{figure}
Once the correct parameters have been displayed, simply call up this line of code, inserting the table cell corresponding 
to the desired format (in this case, cell 0) :
\begin{verbatim}
    cam.set_pixel_format(opencv_formats[0])
\end{verbatim}
Once the connection to the camera has been made and the format has been set, our program will need to take images.
To do this, simply call the following function:
\begin{verbatim}
    frame = cam.get_frame().as_opencv_image()
\end{verbatim}
This function can be modified to suit individual requirements, and loops can be added to capture several images in succession.
%====================================================================================================
% The software
%====================================================================================================
\newpage
\section{The software}
\subsection{Principle}
\begin{figure}[H]
    \centering
    \includegraphics[scale=0.85]{assets/figures/Software/General.png}
    \caption{Software principle as seen by the user}
    \label{fig:Soft_General}
\end{figure}
\subsection{Initialisation}
\begin{figure}[H]
    \centering
    \includegraphics[scale=0.85]{assets/figures/Software/Initialization.png}
    \caption{Block diagram of the initialization function}
    \label{fig:Soft_Init}
\end{figure}
\subsection{Mesure}
blblalbalblabl
\subsection{User interface}
balbalblaba

\chapter{Global system}
\begin{figure}[H]
    \centering
    \includegraphics[scale=0.1]{assets/figures/System/IMG_5308.JPG}
    \caption{Mounted system}
    \label{fig:GLOB_Syst}
\end{figure}
\newpage
\section{System reports}
BLABLABLALBA
\section{Instruction manual}
All the steps required to prepare the measurement are listed below. Steps 7 to 9 are not feasible at present, 
as the software has not yet been implemented in the user interface.
for the time being, as the software has not yet been implemented in the user interface. However, they can be replaced by capturing images with Vimba software. However, we have chosen to note them for future improvements.
\begin{enumerate}
    \item Install the ocular holder fitted with this on the telescope.
    \item Point the telescope at the star and position your eye against the occular.
    \item Adjust the telescope's focal length to see a sharp image.
    \item Installing the assembly (without mask).
    \item Connect the camera to the computer and, if necessary, check the image using Vimba software. 
    If the image is not sharp, adjust the lens so that the sensor is at its focal length.
    \item Insert mask into mask holder.
    \item Start "DIMM V1" software.
    \item Define the document where the results file will be stored and, if necessary, test the camera connection.
    \item Start the measurement and wait for the popup window to display the result.
\end{enumerate}
If the results are inconclusive or faulty, you can modify the camera parameters directly from the interface.
Image and data feedback is also available in this software.

\chapter{Measurement}
This section describes the measurements and image processing methods used to obtain the results.
\section{Image processing}\label{sec:MES_TIS}
First, the software needs to retrieve the camera image.
\begin{figure}[H]
    \centering
    \includegraphics[scale=0.75]{assets/figures/MesuresResultats/ImageSimple.png}
    \caption{Image sent by the camera}
    \label{fig:MES_Ima1}
\end{figure}
An image is recovered using the camera (figure \ref{fig:MES_Ima1}). This image was taken on an "artificial" star, as the weather
was poor during the test phase. The two stars are easy to recognize.
\newpage
Next, a series of 100 images were recovered.
Once all 100 images had been taken, a visual check was carried out to validate the movement of the stars on the screen.
As the background disturbed the basic image (images taken during the day), it was decided to crop it around the
2 stars to remove the bright part (bottom left). The result is shown in Figure \ref{fig:MES_ImaMoy}.
\begin{figure}[H]
    \centering
    \includegraphics[scale=0.95]{assets/figures/MesuresResultats/ImageMoyenne.png}
    \caption{Addition of all images and star location}
    \label{fig:MES_ImaMoy}
\end{figure}
Figure \ref{fig:MES_ImaMoy} also shows the image used to initialize the system. It contains the 100 images summed and rescaled in values from 0 to 255.
The next step is to locate the 2 stars and find their centroid.
Once initialization is complete, the 2 centroids will be stored in memory, along with the star radii, so that they can
be differentiated.
\newpage
\section{Measurement}
The steps described above (Section \ref{sec:MES_TIS}) will then be carried out for each measurement image.
The stars must be located and their centroids determined (as shown in Figure \ref{fig:MES_ImaMoy}).
\begin{figure}[H]
    \centering
    \includegraphics[scale=0.75]{assets/figures/MesuresResultats/BlobsInit.png}
    \caption{Image recognition}
    \label{fig:MES_ImaBlob}
\end{figure}
Each Centroid will be stored in a table specific to the corresponding star.
\newline
Recognition may not work correctly. In this case, the measurement is simply deleted.
However, it would be interesting to add an anomaly detection algorithm later on.
\newpage
\section{Results}
Once all the centroids have been recovered, simply convert them into micrometers to see how the star moves in relation to the turbulence.
\newline
The results of the variations according to the coordinates of the measurements are shown in Figure \ref{fig:MES_VarIm1} and Figure \ref{fig:MES_VarIm2}.
\begin{figure}[H]
    \centering
    \includegraphics[scale=0.6]{assets/figures/MesuresResultats/VariationImage1.png}
    \caption{Coordinate variation (Image 1)}
    \label{fig:MES_VarIm1}
\end{figure}

\begin{figure}[H]
    \centering
    \includegraphics[scale=0.6]{assets/figures/MesuresResultats/VariationImage2.png}
    \caption{Coordinate variation (Image 2)}
    \label{fig:MES_VarIm2}
\end{figure}
\newpage
It's also interesting to represent the centroid of the 2 stars according to their position on the screen. 
In the 2 figures below (\ref{fig:MES_VarCenter1} and \ref{fig:MES_VarCenter2}), the initialization centroids are 
shown in blue and the measurement centroids in red.
\begin{figure}[H]
    \centering
    \includegraphics[scale=0.65]{assets/figures/MesuresResultats/VariationCenter1.png}
    \caption{Shifting centers of gravity (Image 1)}
    \label{fig:MES_VarCenter1}
\end{figure}

\begin{figure}[H]
    \centering
    \includegraphics[scale=0.65]{assets/figures/MesuresResultats/VariationCenter2.png}
    \caption{Shifting centers of gravity (Image 2)}
    \label{fig:MES_VarCenter2}
\end{figure}
The measurements of the centroids in the second image have a value that seems out of line. It will therefore not be analyzed 
in the following section (\ref{sec:ANAL_Pos}). 
These errors may occur despite the protection implemented. 
This is due to the image processing and star recognition algorithms.
This algorithm will be modified to make it more reliable in the future. As it stands, however, it still contains a few bugs.

\chapter{Analysis}
\section{Qualitative analysis}\label{sec:ANAL_Pos}
As far as the mechanical and optical aspects of the project are concerned, the expected results are conclusive.
The mechanical part is usable, works properly and the system is adjustable.
As far as the optical part is concerned, after the preliminary tests carried out, the system is operating correctly.
The result shown in Figure \ref{fig:MES_Ima1} is the one expected and observed for all the measurements made.
\bigbreak
As for the measurement results, the graphs in figures \ref{fig:MES_VarCenter1} and \ref{fig:MES_VarCenter2} show that star
movements have been detected.
The displacement zone is also around the initialization centroid.
These results were expected and are very encouraging.
\newline
The next step is to calculate the standard deviation of these results, convert it into an angle (arc second) and output the associated
Fried parameter. Unfortunately, due to lack of time, this part has not yet been completed.
\newline
\textbf{\textcolor{red}{RAJOUTER ICI QQCH}}
\section{Quantitative analysis}
As things stand, enough can be done to achieve a conclusive result. However, image processing and selection methods could be optimized.
This optimization requires time and thorough testing with a large batch of measurements (as many different ones as possible).
\newline
Some measurements are still recognized as good when in fact they are not (Figure \ref{fig:MES_VarCenter2}).
It would therefore be interesting to be able to add anomaly detection to the measurements so as to avoid such results. \newline
However, the software delivered the expected results. Star detection and foreground processing are sufficient for preliminary tests.
It would be interesting to test a batch of images (e.g. 500) by splitting them into several parts to find out the accuracy limit
of the measurement.
\newpage
\section{Achievements and improvements}
\subsection{Achievements}
For the mechanical and optical parts, all the expected results were achieved. These results are listed below:
\begin{itemize}
    \item Optical development of the \Gls{DIMM} system
    \item Mechanical development of the \Gls{DIMM} system
    \item ZEMAX and SOLIDWORKS design of the system.
    \item Preparation of drawings for machining mechanical components.
    \item Design of the 12" MEADE telescope for the optics laboratory.
\end{itemize}
For the software part, the following objectives were achieved:
\begin{itemize}
    \item Creation of an image analysis program.
    \item Detection of the image area of interest.
    \item Separation of the 2 images.
    \item Centroid recovery for each image.
    \item Data retrieval and display.
\end{itemize}
\subsection{Improvements}
The software works correctly. However, there are a few improvements to be made and optimizations to be added once
a larger data set has been obtained. \newline
The enhancements required to keep the system running smoothly are listed below:
\begin{itemize}
    \item Detection of measurement anomalies.
    \item Image processing optimization.
    \item More reliable image processing (requires a large dataset).
    \item Addition and modification of the alghorithm for sunspot detection and seeing profil.
    \item Added alghorithm for transforming data into user-understandable values.
    \item Perform tests with different dataset sizes to find out the system's accuracy limits.
\end{itemize}
Improvements that would be accessory to the proper functioning of the system, but which would greatly help it and make it more reliable, are :
\begin{itemize}
    \item Measurement alghoritms added to user interface.
    \item Addition of functions necessary for user understanding of data analysis.
          (For example: adding graphs, tracking measurements, saving measurements and vision parameters)
    \item Addition of pre-implemented measurement parameters.
          (After several tests, it would be interesting to pre-program measurement profiles for each measurable star).
\end{itemize}

\chapter{Conclusion}
In this thesis, a \Gls{DIMM} system was dimmed and adapted to the \Gls{heig-vd} laboratory telescope.
The aim of this project is to help the institute \Gls{IRSOL} so that they can determine whether the use of
an adaptative optics (\Gls{ao}) system is viable or not. This project, in addition to another,
will be extremely useful for the institute's future development.
\bigbreak
The work carried out during this thesis was :
\begin{itemize}
    \item Produce an optical design of the system.
    \item Produce the mechanical design for adaptation to the laboratory telescope.
    \item Prepare manufacturing drawings.
    \item Manufacture and assemble the system.
    \item Create software for measuring turbulence.
    \item On-site testing.
\end{itemize}
\bigbreak
The results are highly conclusive. \newline
The optical and mechanical designs are functional and tested on several stars (real and virtual). \newline
Star movements due to turbulence are measurable with the software and do not appear aberrant.
However, it would be nice to be able to convert the data into a Fried parameter value.
This part could not be realized due to lack of time, but does not require much development time.
\newline
As far as the software is concerned, V1 is functional and gives good results.
In addition, a Windows application has been created to make the software easier to understand for the user.
It would be interesting to make several optimizations, but further tests on several stars are necessary.
\bigbreak
To conclude this thesis, the results are conclusive and most of the objectives have been achieved.
A \Gls{DIMM} system has been built and mounted on the laboratory telescope, and tests have been carried out.
The results are not yet in their final shape, but promise good system efficiency.
The \Gls{IRSOL} site can't use this project for development purposes just yet, but it's only a small step away
from knowing the quality of their observation site.\newline
This system could be useful for more than just IRSOL. In fact, the system could also be used to locate a good site for 
an observatory, to study the evolution of turbulence and, perhaps one day, to forecast it.
\newpage
\vfil
\hspace{8cm}\makeatletter\@author\makeatother\par
\hspace{8cm}\begin{minipage}{5cm}
    % Place pour signature numérique
    \printsignature
\end{minipage}

\clearpage
% \printbibliography
\chapter*{Bibliography}
\addcontentsline{toc}{chapter}{Bibliography}
|1| L.Jolissaint, "Optique adaptative au foyer d'un télescope de la classe 1 mètre", M.S. thesis, Geneva: 2001, 2001.
\bigbreak
|2| IRSOL, "Research activity: Research activity at irsol", [Online]. \newline Available: \hyperlink{IRSOL website}{https://www.irsol.usi.ch/research/research-activity/}. (accessed: 24.07.2023).
\bigbreak
|3| Sarazin, M., & Roddier, F., "The ESO differential image motion monitor", Astronomy and Astrophysics, 1990, (pages 294 to 300).
\bigbreak
|4| Edmund optics, [Online]. Available: \hyperlink{Edmund optics website}{https://www.edmundoptics.eu/}. (accessed: 17.07.2023).
\bigbreak
|5| Astroshop, "MAEDE solar filter 1375", [Online]. Available: \hyperlink{Astroshop website}{https://www.astroshop.de/fr/filtres-en-verre-montes/meade-filtre-solaire-1375-id-349-mm/p,59498}. (accessed: 17.07.2023).
\bigbreak
|6| L.Jolissaint, "Introduction à l'optique pour les ingénieurs", Course theory, HEIG-VD: 2023, 2023.

\label{glossaire}
\printnoidxglossary
\label{index}
\printindex

\appendix
\appendixpage
\addappheadtotoc

%====================================================================================================
% MEP Perso
%====================================================================================================
\chapter{Personnal drawing}
% ============================================ Adapter ==============================================
\label{App:MEP}
\includepdf[pages=-,angle=90,scale =0.9]{assets/Annexes/MEP/Adaptateur - Feuille1.pdf}
\newpage
% ========================================= Occulare screw ==========================================
\includepdf[pages=-,angle=90,scale =0.9]{assets/Annexes/MEP/Bague_Objectif - Feuille1.pdf}
\newpage
% ========================================= Occulare screw ==========================================
\includepdf[pages=-,angle=90,scale =0.9]{assets/Annexes/MEP/Masque - Feuille1.pdf}
\newpage
% ========================================= Occulare screw ==========================================
\includepdf[pages=-,angle=90,scale =0.9]{assets/Annexes/MEP/SupportMasque1 - Feuille1.pdf}
\newpage
% ========================================= Occulare screw ==========================================
\includepdf[pages=-,angle=90,scale =0.9]{assets/Annexes/MEP/SupportMasque2 - Feuille1.pdf}
\newpage
% ========================================= Occulare screw ==========================================
\includepdf[pages=-,angle=90,scale =0.9]{assets/Annexes/MEP/SoclePrisme1 - Feuille1.pdf}
\newpage
% ========================================= Occulare screw ==========================================
\includepdf[pages=-,angle=90,scale =0.9]{assets/Annexes/MEP/SoclePrisme2 - Feuille1.pdf}
\newpage
% ========================================= Occulare screw ==========================================
\includepdf[pages=-,angle=90,scale =0.9]{assets/Annexes/MEP/MaintientPrisme - Feuille.pdf}
\newpage
% ========================================= Occulare screw ==========================================
\includepdf[pages=-,angle=90,scale =0.9]{assets/Annexes/MEP/SupportCamera - Feuille1.pdf}
\newpage

%====================================================================================================
% EDMUND MEP
%====================================================================================================
\chapter{Edmund optics components drawing} \label{App:Edmund_MEP}
% ========================================= Occulare screw ==========================================
\includepdf[pages=-,angle=90,scale =0.9]{assets/Annexes/MEP Edmund/Prism.pdf}
\newpage
% ========================================= Occulare screw ==========================================
\includepdf[pages=-,angle=90,scale =0.9]{assets/Annexes/MEP Edmund/Lens.pdf}
\newpage
% ========================================= Occulare screw ==========================================
\includepdf[pages=-,angle=90,scale =0.9]{assets/Annexes/MEP Edmund/Support_Lens.pdf}
\newpage
% ========================================= Occulare screw ==========================================
\includepdf[pages=-,angle=90,scale =0.9]{assets/Annexes/MEP Edmund/Support_Lens2.pdf}
\newpage
% ========================================= Occulare screw ==========================================
\includepdf[pages=4,scale =0.9]{assets/Annexes/MEP Edmund/Camera.pdf}
\newpage

%====================================================================================================
% Other components
%====================================================================================================
\chapter{Datasheet of all other components}
% ========================================= Occulare screw ==========================================
\includepdf[pages=1-3,scale =0.9]{assets/Annexes/MEP Edmund/Camera.pdf}
\newpage

%====================================================================================================
% Matlab code
%====================================================================================================
\chapter{Matlab Code} \label{App:Matlab}
% ========================================= Occulare screw ==========================================
\includepdf[pages=-,scale =0.9]{assets/Annexes/Matlab/Code matlab Ray_trace.pdf}
\newpage
% ========================================= Occulare screw ==========================================
\includepdf[pages=-,scale =0.9]{assets/Annexes/Matlab/Results of matlab Programm for ray tracing.pdf}
\newpage

% Le colophon est le dernier élément d'un document qui contient des notes de l'auteur concernant la mise en page et l'édition du document : il est parfaitement optionnel.
% \input{colophon.tex}

\end{document}
