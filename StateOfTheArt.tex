The \Gls{DIMM} concept is well known in the world of astrophysics, and has already been implemented in several forms.
However, most of the \Gls{DIMM}s produced refer to article "The ESO differential image motion monitor" |3|.
Mr Sarazin and Mr Roddier have produced a seeing monitor for ESO and describe it in this article.

\section{Principle}
The principle is the same as that described in section \ref{sec:Theo_DIMM}, with the light beam passing through a turbulent layer
before reaching the telescope. This beam is then masked, deflected by a prism and finally focused on a screen to produce
2 distinct images. In the next section, we'll be working out the seeing measurement algorithm. All the calculations involved
in determining seeing are listed and explained in ESO's article on DIMM |3|.
\section{Realization}
When it comes to \Gls{DIMM} realizations, there are several variants :
\begin{itemize}
    \item Mask installed in front of the telescope pupil.
    \item Mask and prisms installed in front of the telescope pupil.
    \item Mask and prisms installed after telescope (Mask on exit pupil).
    \item Prism installed only for one of the masked beams.
    \item Prism installed for 2 masked beams.
\end{itemize}
All these solutions have achieved results, but the methods of image processing and acquisition are never the same.
\newpage
\section{Alternative methods}
There are several ways of measuring seeing other than \Gls{DIMM}. Some of these methods are listed below :
\begin{itemize}
    \item \textbf{Lucky imaging} : Lucky Imaging takes advantage of the fact that atmospheric turbulence changes
          rapidly over time. The technique involves capturing a large series of images very quickly,
          usually in a few milliseconds. This high frame rate enables us to freeze moments of less severe atmospheric
          turbulence and to reject images captured during moments of high turbulence. Once a clear image of the object
          has been recovered, instant seeing can be determined by taking further successive images, using the image
          obtained by Lucky imaging as a reference.
    \item \textbf{LSIM (Least-Squares Image Motion)} : Unlike Lucky Imaging, which selects lucky images, the LSIM
          method uses all the images captured in the sequence to calculate a single global turbulence displacement,
          which it then corrects to improve the quality of the final image. As with the LSIM method, the seeing can then
          be determined using the reference image found with this method.
    \item \textbf{Scintillation Analysis} : This method is more similar to \Gls{DIMM}. However, the sensor used is a photometer.
          This device is used to record the star's brightness at regular time intervals. This allows us to capture variations
          in brightness due to flicker. The flicker's characteristics are compared with theoretical atmospheric models
          to obtain the current seeing.
\end{itemize}