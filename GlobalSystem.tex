\begin{figure}[H]
    \centering
    \includegraphics[scale=0.1]{assets/figures/System/IMG_5308.JPG}
    \caption{Mounted system}
    \label{fig:GLOB_Syst}
\end{figure}
\newpage
\section{System reports}
BLABLABLALBA
\section{Instruction manual}
All the steps required to prepare the measurement are listed below. Steps 7 to 9 are not feasible at present, 
as the software has not yet been implemented in the user interface.
for the time being, as the software has not yet been implemented in the user interface. However, they can be replaced by capturing images with Vimba software. However, we have chosen to note them for future improvements.
\begin{enumerate}
    \item Install the ocular holder fitted with this on the telescope.
    \item Point the telescope at the star and position your eye against the occular.
    \item Adjust the telescope's focal length to see a sharp image.
    \item Installing the assembly (without mask).
    \item Connect the camera to the computer and, if necessary, check the image using Vimba software. 
    If the image is not sharp, adjust the lens so that the sensor is at its focal length.
    \item Insert mask into mask holder.
    \item Start "DIMM V1" software.
    \item Define the document where the results file will be stored and, if necessary, test the camera connection.
    \item Start the measurement and wait for the popup window to display the result.
\end{enumerate}
If the results are inconclusive or faulty, you can modify the camera parameters directly from the interface.
Image and data feedback is also available in this software.