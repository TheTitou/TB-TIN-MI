%%
This chapter introduces the context and generalities relative to the work.



%====================================================================================================
% Context
%====================================================================================================
\section{Context}
The Institute for Solar Reasearch (\Gls{IRSOL}) can be found in Locarno Switzerland. At the time of writing and according to the observatory, their reasearch activities focus on :
<<
\begin{enumerate}
    \item observational spectropolarimetry and instrument development,
    \item theoretical modeling of the generation and transfer of polarized radiation,
    \item numerical simulation of the solar atmosphere and numerical radiative transfer.
\end{enumerate}>>\footnote{\cite{IRSOL_research} IRSOL, 2023. RESEARCH ACTIVITY, Research Activity at IRSOL. [Online]. [Accessed on 24.03.2023]. Available at: https://www.irsol.usi.ch/research/research-activity/}
In odrer to asses the usefullness of adaptative optics two Bachelor thesis are conducted.
\begin{enumerate}
    \item "Analyseur de la turbulence optique pour le télescope solaire IRSOL" (Optical turbulence analyser for \Gls{IRSOL}'s solar telescope)
    \item "Le mien" (this thesis)
\end{enumerate}
The 1\textsuperscript{st} one aims to analyse optical turbulence on site. The 2\textsuperscript{nd} one is meant to measure the optical imperfections of the telescope itsef and maybe of other optical equipment.

Once both studies are concluded, the laboratory will be able to determine whether the use of \acrfull{ao} makes sense or not. Should teh atmospheric turbulencec have a greater impact on the image than the telescope's imperfections, A nes study on the implementation of \acrshort{ao} will be conducted. This study is out of the scope of either thesis.



%====================================================================================================
% Issue at hand
%====================================================================================================
\section{Issue at hand}

The issue at hand is that the the teslescope has some unkown optical imperfecitons. This study aims to measure them.

The short methodology explanation is as follows:
\begin{enumerate}
    \item Sets of 2 pictures are taken (1 focused, 1 unfocused (in a known way))
    \item The best sets are kept and stacked.
    \item The stack is run throug a code which output will then be interpreted to find the imperfections.
\end{enumerate}

If the exposure time is short enough, turbulences will eventually be fixed into place, resulting in a sharp image.\footnote{\cite{Law_lucky} Law, 2006. Lucky imaging: High angular resolution imaging in the visible from the ground. in: Astronomy \& Astrophysics [online], February 2006, N°446, pp. 739-745. ISSN
1950-6295. [Accessed on 24.02.2023]. Available at: https://www.aanda.org/articles/aa/abs/2006/05/} Furthermore, if the time between a set of 2 pictures is short enough, the perturbation will not have moved which means a set has the same perturbations for both pictures.

In a perfect world, the exposures time is infinitely small and the time inbetween two frames inexistent. This will be further discussed.
The process of stacking images is called lucky imaging. The theory behind it it that the mean turbulence is null\footnote{\cite{JOLISSAINT_master} JOLISSAINT, Laurent, 1996. XXXXX. Genève : XXXX. Master's thesis}, thus the result should be a picture free of aberations caused by the atmosphere.

Alternatively, long exposure times will also cancel the effect of atmospheric turbulence.

The long explanation is the subject of this document.

