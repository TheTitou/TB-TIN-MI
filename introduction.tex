This chapter introduces the context and generalities relative to the work.

%====================================================================================================
% Context
%====================================================================================================
\section{Context}
This bachelor's thesis was carried out to support the research of the institute \Gls{IRSOL}. \newline
<<\Gls{IRSOL} is a research institute dependent on a foundation of the same name and affiliated to Università della Svizzera italiana.>>$^1$
The institute is based above Locarno in Ticino. \newline
Their research is mainly based on the measurement and characterization of the sun, and specializes in :
<<
\begin{enumerate}
    \item observational spectropolarimetry and instrument development,
    \item theoretical modeling of the generation and transfer of polarized radiation,
    \item numerical simulation of the solar atmosphere and numerical radiative transfer.
\end{enumerate}>>\footnote{\cite{IRSOL_research} \Gls{IRSOL}, 2023. RESEARCH ACTIVITY, Research Activity at \Gls{IRSOL}. [Online]. [Accessed on 21.07.2023]. Available at: https://www.irsol.usi.ch/research/research-activity/}

The problem is that the institute is looking to expand its equipment in order to develop their research.
However, it is necessary to know the limitations of the measuring instruments, as well as the quality of the observation site.
Their first development objective would be to integrate an \acrfull{ao} system.
\newline
In odrer to asses the usefullness of \Gls{ao} two Bachelor thesis are conducted.
\begin{enumerate}
    \item "Caractérisation du télescope de l'observatoire solaire \Gls{IRSOL} (Tessin)" (Characterization of the \Gls{IRSOL} solar observatory telescope (Ticino))
    \item This thesis : "Analyseur de la turbulence optique pour le télescope solaire \Gls{IRSOL}" (Optical turbulence analyser for \Gls{IRSOL}'s solar telescope)
\end{enumerate}
The 1\textsuperscript{st} one is meant to measure the optical imperfections of the telescope itself and maybe of other optical equipment.
The 2\textsuperscript{nd} one aims to analyse optical turbulence on site.

Once these two projects are completed, \Gls{IRSOL} will be able to determine whether the acquisition of an \acrfull{ao} makes sense
and they will also be able to quantify the limiting element to their research (the observation site or the telescope).



%====================================================================================================
% Issue at hand
%====================================================================================================
\section{Issue at hand}
In a discussion with the institute, it was said: "Seeing evolves over the course of the day.
When the sun rises, the ground warms up and the heat rises. This is when seeing is at its worst.
But when the sun goes down, there's a moment when seeing is almost non-existent because the air currents freeze before going down again."
\bigbreak
The issue at hand is that the quality of the observation site is known but cannot be quantified.
The aim of this study is to quantify the quality of their observation site.
The instrument used will be the \Gls{DIMM} (Differential image motion monitor).

A \Gls{DIMM} works as follows (simplified version) :
\begin{enumerate}
    \item Capture an image masked by two small holes.
    \item Tracking the movement of these images.
    \item Measurement of the standard deviation of this movement.
    \item Conversion of previous result into turbulent layer thickness.
\end{enumerate}
The aim will be to play with the exposure time to determine the location of the image center, but also to freeze the turbulence.
\newline
Atmospheric disturbances will vary the centroid of the image, and it is this element that will be measured.
By taking the standard deviation of all these measurements, it is possible to give a result that qualifies the disturbance in a defined time.
\newline
The long explanation is the subject of this document.

