\section{Qualitative analysis}\label{sec:ANAL_Pos}
As far as the mechanical and optical aspects of the project are concerned, the expected results are conclusive.
The mechanical part is usable, works properly and the system is adjustable.
As far as the optical part is concerned, after the preliminary tests carried out, the system is operating correctly.
The result shown in Figure \ref{fig:MES_Ima1} is the one expected and observed for all the measurements made.
\bigbreak
As for the measurement results, the graphs in figures \ref{fig:MES_VarCenter1} and \ref{fig:MES_VarCenter2} show that star
movements have been detected.
The displacement zone is also around the initialization centroid.
These results were expected and are very encouraging.
\newline
The next step is to calculate the standard deviation of these results, convert it into an angle (arc second) and output the associated
Fried parameter. Unfortunately, due to lack of time, this part has not yet been completed.
\newline
\textbf{\textcolor{red}{RAJOUTER ICI QQCH}}
\section{Quantitative analysis}
As things stand, enough can be done to achieve a conclusive result. However, image processing and selection methods could be optimized.
This optimization requires time and thorough testing with a large batch of measurements (as many different ones as possible).
\newline
Some measurements are still recognized as good when in fact they are not (Figure \ref{fig:MES_VarCenter2}).
It would therefore be interesting to be able to add anomaly detection to the measurements so as to avoid such results. \newline
However, the software delivered the expected results. Star detection and foreground processing are sufficient for preliminary tests.
It would be interesting to test a batch of images (e.g. 500) by splitting them into several parts to find out the accuracy limit
of the measurement.
\newpage
\section{Achievements and improvements}
\subsection{Achievements}
For the mechanical and optical parts, all the expected results were achieved. These results are listed below:
\begin{itemize}
    \item Optical development of the \Gls{DIMM} system
    \item Mechanical development of the \Gls{DIMM} system
    \item ZEMAX and SOLIDWORKS design of the system.
    \item Preparation of drawings for machining mechanical components.
    \item Design of the 12" MEADE telescope for the optics laboratory.
\end{itemize}
For the software part, the following objectives were achieved:
\begin{itemize}
    \item Creation of an image analysis program.
    \item Detection of the image area of interest.
    \item Separation of the 2 images.
    \item Centroid recovery for each image.
    \item Data retrieval and display.
\end{itemize}
\subsection{Improvements}
The software works correctly. However, there are a few improvements to be made and optimizations to be added once
a larger data set has been obtained. \newline
The enhancements required to keep the system running smoothly are listed below:
\begin{itemize}
    \item Detection of measurement anomalies.
    \item Image processing optimization.
    \item More reliable image processing (requires a large dataset).
    \item Addition and modification of the alghorithm for sunspot detection and seeing profil.
    \item Added alghorithm for transforming data into user-understandable values.
    \item Perform tests with different dataset sizes to find out the system's accuracy limits.
\end{itemize}
Improvements that would be accessory to the proper functioning of the system, but which would greatly help it and make it more reliable, are :
\begin{itemize}
    \item Measurement alghoritms added to user interface.
    \item Addition of functions necessary for user understanding of data analysis.
          (For example: adding graphs, tracking measurements, saving measurements and vision parameters)
    \item Addition of pre-implemented measurement parameters.
          (After several tests, it would be interesting to pre-program measurement profiles for each measurable star).
\end{itemize}