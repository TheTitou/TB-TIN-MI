%====================================================================================================
% Possibilities and choices
%====================================================================================================
\section{Possibilities and choices}
\subsection{Possibilities}
\subsubsection{measurement method}
Two options are available for creating the software:
\begin{enumerate}
    \item \textbf{Pre-capture of images using Vimba software (\ref{sec:Vimba})} : \newline
          In this case, the images are taken using the software supplied with the camera and then entered into the software develop with python.
          The problem is that the operator carrying out the measurement will have to switch between initialization parameters and parameters
          for measurements on the Vimba software. For this solution, different software modes will also be required
    \item \textbf{Image capture and analysis} :\newline
          The second solution involves automatic measurement and initialization. The operator would simply point the telescope at an
          object and launch the software. The problem with this solution is that it would be more complex to detect a bug. In this case,
          we'd have to come up with solutions to recognize measurement bugs as quickly as possible.
\end{enumerate}
\subsubsection{Initialization}
For initialization, two solutions are also possible :
\begin{enumerate}
    \item The first is to take several images with a exposure time equal to the measurement part. By adding up all the images,
          the turbulence will be averaged. It will then be easy to determine the object's center of gravity and its centroid.
    \item The second option would be to considerably increase the exposure time of the image. This would be the same as the
          previous option, but taking only 1 image would reduce memory usage and increase computing speed.
\end{enumerate}
In these 2 cases, the number of images to be used or the exposure time required to correctly average the image will need to
be determined during the initial tests. If turbulence is not properly averaged, aberrant results may appear.
\subsubsection{Measurement}
bablablalba
\subsection{Choices}
\subsubsection{measurement method}
balbalblaba
\subsubsection{Initialization}
blalbalblalb
\subsubsection{Measurement}
bablablalba
%====================================================================================================
% Vimba software
%====================================================================================================
\newpage
\section{Vimba software}\label{sec:Vimba}
The Vimba software package was created by Allied Vision and is available to all users of Allied Vision cameras.
The software lets you modify camera parameters in real time (Figure \ref{fig:Soft_Vimba}). It is also designed to pre-process
the image before sending it for output. Inputs are also available to control the camera.
\begin{figure}[H]
    \centering
    \includegraphics[scale=0.85]{assets/figures/Software/VimbaInterface.png}
    \caption{Editable image parameters on Vimba}
    \label{fig:Soft_Vimba}
\end{figure}
This application is very useful for adjusting parameters during measurements, but also for initial tests, as several images can
be taken in quick succession using one of the software's functions (With the "Image series options" parameter).
%====================================================================================================
% Python API for Vimba
%====================================================================================================
\newpage
\section{Python API for Vimba}
The Vimba software introduced in section \ref{sec:Vimba} is also supplied with an API for simulating the software on python.
All API data is given in a git directory (\href{https://github.com/alliedvision/VimbaPython}{Github link}) and the user manual
can be found in appendix \ref{App:pythonVimba}.\newline
Please note that if the Vimba program is open on the computer, the API will not be able to communicate with the camera.
\subsection{Camera connection}\label{sec:Soft_API_Connect}
Before using the camera with the API, the code must call the procedure that simulates the Vimba program :
\begin{verbatim}
    with Vimba.get_instance() as vimba:
        cams = vimba.get_all_cameras()
        with cams[0] as cam:
            // Your code
\end{verbatim}
In the code above, the "cam" structure is the element to be called in order to access all the camera parameters.
With this structure, image parameters, setups and access to images can be carried out.
\subsection{Parameter modification}
The API also lets you change camera parameters separately. In the case of the dimm application, the most important
need is to change the exposure time. To do this, once the camera is connected (Code in section \ref{sec:Soft_API_Connect}),
call this function:
\begin{verbatim}
    exposure_time.set(2000) #us
\end{verbatim}
This function lets you change the exposure time (given in microseconds) and can range from 20 to 10000000 (10 seconds).
\subsection{Load and save settings}
\subsubsection{Save settings}
Pre-recorded settings files can be created for the camera. If certain values are modified and you wish to keep them,
you can save the file on your computer. To do this, after connecting to the camera (\ref{sec:Soft_API_Connect}),
simply call up the function :
\begin{verbatim}
    settings_file = 'my_File.xml'
    cam.save_settings(settings_file, PersistType.All)
\end{verbatim}
\subsubsection{Load settings}
To retrieve settings that have already been saved and load them into the camera, use the code below:
\begin{verbatim}
    settings_file = 'my_File.xml'
    cam.load_settings(settings_file, PersistType.All)
\end{verbatim}
\newpage
\subsection{Image capture}
Before taking an image, you need to be able to set a format compatible with the compiler or 
a library that will be used to create the software. To do this, the following code compares 
the camera's available formats with the compatible formats.
\begin{verbatim}
    with Vimba.get_instance() as vimba:
        cams = vimba.get_all_cameras()
        with cams[0] as cam:
            formats = cam.get_pixel_formats()
            opencv_formats = intersect_pixel_formats(formats, OPENCV_PIXEL_FORMATS)
        print(f"Available formats:")
        for i, format in enumerate(formats):
            print(i, format)
        print(f"\nOpencv compatible formats:")
        for i, format in enumerate(opencv_formats):
            print(i, format)
\end{verbatim}
\begin{figure}[H]
    \centering
    \includegraphics[scale=0.85]{assets/figures/Software/Format_Available.png}
    \caption{Compatible format results}
    \label{fig:Soft_Vimba_Format}
\end{figure}
Once the correct parameters have been displayed, simply call up this line of code, inserting the table cell corresponding 
to the desired format (in this case, cell 0) :
\begin{verbatim}
    cam.set_pixel_format(opencv_formats[0])
\end{verbatim}
Once the connection to the camera has been made and the format has been set, our program will need to take images.
To do this, simply call the following function:
\begin{verbatim}
    frame = cam.get_frame().as_opencv_image()
\end{verbatim}
This function can be modified to suit individual requirements, and loops can be added to capture several images in succession.
%====================================================================================================
% The software
%====================================================================================================
\newpage
\section{The software}
\subsection{Principle}
\begin{figure}[H]
    \centering
    \includegraphics[scale=0.85]{assets/figures/Software/General.png}
    \caption{Software principle as seen by the user}
    \label{fig:Soft_General}
\end{figure}
\subsection{Initialisation}
\begin{figure}[H]
    \centering
    \includegraphics[scale=0.85]{assets/figures/Software/Initialization.png}
    \caption{Block diagram of the initialization function}
    \label{fig:Soft_Init}
\end{figure}
\subsection{Mesure}
blblalbalblabl
\subsection{User interface}
balbalblaba