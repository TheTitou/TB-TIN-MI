In this thesis, a \Gls{DIMM} system was dimmed and adapted to the \Gls{heig-vd} laboratory telescope.
The aim of this project is to help the institute \Gls{IRSOL} so that they can determine whether the use of
an adaptative optics (\Gls{ao}) system is viable or not. This project, in addition to another,
will be extremely useful for the institute's future development.
\bigbreak
The work carried out during this thesis was :
\begin{itemize}
    \item Produce an optical design of the system.
    \item Produce the mechanical design for adaptation to the laboratory telescope.
    \item Prepare manufacturing drawings.
    \item Manufacture and assemble the system.
    \item Create software for measuring turbulence.
    \item On-site testing.
\end{itemize}
\bigbreak
The results are highly conclusive. \newline
The optical and mechanical designs are functional and tested on several stars (real and virtual). \newline
Star movements due to turbulence are measurable with the software and do not appear aberrant.
However, it would be nice to be able to convert the data into a Fried parameter value.
This part could not be realized due to lack of time, but does not require much development time.
\newline
As far as the software is concerned, V1 is functional and gives good results.
In addition, a Windows application has been created to make the software easier to understand for the user.
It would be interesting to make several optimizations, but further tests on several stars are necessary.
\bigbreak
To conclude this thesis, the results are conclusive and most of the objectives have been achieved.
A \Gls{DIMM} system has been built and mounted on the laboratory telescope, and tests have been carried out.
The results are not yet in their final shape, but promise good system efficiency.
The \Gls{IRSOL} site can't use this project for development purposes just yet, but it's only a small step away
from knowing the quality of their observation site.
\newpage
\vfil
\hspace{8cm}\makeatletter\@author\makeatother\par
\hspace{8cm}\begin{minipage}{5cm}
    % Place pour signature numérique
    \printsignature
\end{minipage}